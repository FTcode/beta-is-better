\documentclass{article}

\usepackage{amsmath}
\usepackage{amsthm}
\usepackage{mathtools}
\usepackage{amsfonts}
\usepackage{url}

\usepackage[colorinlistoftodos]{todonotes}
\newcommand{\filip}[1]{\todo[size=\small,linecolor=violet, backgroundcolor=violet!30!white,bordercolor=violet, textcolor=black]{Filip: #1}}
\newcommand{\paola}[1]{\todo[size=\small,linecolor=blue!40!white, backgroundcolor=blue!20!white,bordercolor=blue]{Paola: #1}}

\newcommand{\EE}{\ensuremath{\mathbb{E}}}
\newcommand{\NN}{\ensuremath{\mathbb{N}}}
\newcommand{\RR}{{\ensuremath{\mathbb{R}}}}
\newcommand{\QQ}{\ensuremath{\mathbb{Q}}}
\newcommand{\ZZ}{\ensuremath{\mathbb{Z}}}
\newcommand{\OO}{\mathcal{O}}
\newcommand{\CC}{{\mathbb{C}}}
\renewcommand{\SS}{{\mathbb{S}}}
\renewcommand{\vec}[1]{\ensuremath{\mathbf{#1}}}

\newtheorem{lemma}{Lemma}
\newtheorem{thm}{Theorem}
\newtheorem{claim}{Claim}
\newtheorem{cor}{Corollary}
\newtheorem{prop}{Proposition}
\theoremstyle{definition}
\newtheorem{defn}{Definition}
\newtheorem{rmk}{Remark}
\newtheorem{eg}{Example}

\DeclarePairedDelimiter\abs{\lvert}{\rvert}%
\DeclarePairedDelimiter\norm{\lVert}{\rVert}%
\DeclarePairedDelimiter{\nearestint}\lfloor\rceil%
\DeclareMathOperator{\rank}{rank}
\DeclareMathOperator{\lgh}{lgh}
\DeclareMathOperator{\slope}{slope}
\DeclareMathOperator{\skewind}{skew\& ind}
\DeclareMathOperator{\erf}{erf}

%% inner-product
\newcommand{\innerP}[2]{\langle #1; #2 \rangle}

%% mathcal
\def\A{{\cal A}}
\def\B{{\cal B}}
\def\C{{\cal C}}
\def\D{{\cal D}}
\def\E{{\cal E}}
\def\e{{\cal e}}
\def\F{{\cal F}}
\def\G{{\cal G}}
\def\H{{\cal H}}
\def\I{{\cal I}}
\def\J{{\cal J}}
\def\K{{\cal K}}
\def\L{{\cal L}}
\def\M{{\cal M}}
\def\N{{\cal N}}
\def\O{{\cal O}}
\def\P{{\cal P}}
\def\Q{{\cal Q}}
\def\R{{\cal R}}
\def\S{{\cal S}}
\def\T{{\cal T}}
\def\U{{\cal U}}
\def\V{{\cal V}}
\def\W{{\cal W}}
\def\X{{\cal X}}
\def\Y{{\cal Y}}
\def\Z{{\cal Z}}

%% vectors
\newcommand{\va}{{\vec{a}}}
\newcommand{\vfka}{{\vec{\fka}}}
\newcommand{\vA}{{\vec{A}}}
\newcommand{\vfkA}{{\vec{\fkA}}}
\newcommand{\vb}{{\vec{b}}}
\newcommand{\vB}{{\vec{B}}}
\newcommand{\vfkB}{{\vec{\fkB}}}
\newcommand{\vfkb}{{\vec{\fkb}}}
\newcommand{\vc}{{\vec{c}}}
\newcommand{\vC}{{\vec{C}}}
\newcommand{\vtlC}{{\vec{\tl{C}}}}
\newcommand{\vchi}{{\vec{\chi}}}
\newcommand{\vD}{{\vec{D}}}
\newcommand{\vDelta}{{\vec{\Delta}}}
\newcommand{\ve}{{\vec{e}}}
\newcommand{\vh}{{\vec{h}}}
\newcommand{\vm}{{\vec{m}}}
\newcommand{\vM}{{\vec{M}}}
\newcommand{\vp}{{\vec{p}}}
\newcommand{\vr}{{\vec{r}}}
\newcommand{\vtlr}{{\vec{\tl{r}}}}
\newcommand{\vfkr}{{\vec{\fkr}}}
\newcommand{\vtlfkr}{{\vec{\tl\fkr}}}
\newcommand{\vR}{{\vec{R}}}
\newcommand{\vs}{{\vec{s}}}
\newcommand{\vS}{{\vec{S}}}
\newcommand{\vT}{{\vec{T}}}
\newcommand{\vu}{{\vec{u}}}
\newcommand{\vU}{{\vec{U}}}
\newcommand{\vfkU}{{\vec{\fkU}}}
\newcommand{\vv}{{\vec{v}}}
\newcommand{\vV}{{\vec{V}}}
\newcommand{\vfkV}{{\vec{\frak{V}}}}
\newcommand{\vw}{{\vec{w}}}
\newcommand{\vW}{{\vec{W}}}
\newcommand{\vfkW}{{\vec{\frak{W}}}}
\newcommand{\vx}{{\vec{x}}}
\newcommand{\vX}{{\vec{X}}}
\newcommand{\vfkx}{{\vec{\fkx}}}
\newcommand{\vfkX}{{\vec{\fkX}}}
\newcommand{\vy}{{\vec{y}}}
\newcommand{\vY}{{\vec{Y}}}
\newcommand{\vfky}{{\vec{\fky}}}
\newcommand{\vfkY}{{\vec{\fkY}}}
\newcommand{\vz}{{\vec{z}}}
\newcommand{\vZ}{{\vec{Z}}}
\newcommand{\vfkz}{{\vec{\fkz}}}
\newcommand{\vzero}{{\vec{0}}}
\newcommand{\vone}{{\vec{1}}}
\newcommand{\vCom}{\vec{\Com}}
\newcommand{\vpi}{\vec{\pi}}
\newcommand{\vPi}{\vec{\Pi}}
\newcommand{\vrho}{\vec{\rho}}
\newcommand{\vvarphi}{{\vec{\varphi}}}
\newcommand{\vpsi}{{\vec{\psi}}}
\newcommand{\vtheta}{{\vec{\theta}}}

% mathfrak
\newcommand{\fka}{\mathfrak{a}}
% \newcommand{\fkA}{\mathfrak{A}}
% \newcommand{\fkalpha}{\mathfrak{\alpha}}
\newcommand{\fkb}{\mathfrak{b}}
\newcommand{\fkB}{\mathfrak{B}}
\newcommand{\fkC}{\mathfrak{C}}
\newcommand{\fkD}{\mathfrak{D}}
\newcommand{\fkg}{\mathfrak{g}}
\newcommand{\fkG}{\mathfrak{G}}
\newcommand{\fkh}{\mathfrak{h}}
\newcommand{\fkH}{\mathfrak{H}}
\newcommand{\fki}{\mathfrak{i}}
\newcommand{\fkI}{{\mathfrak{I}}}
\newcommand{\fkn}{\mathfrak{n}}
\newcommand{\fkr}{\mathfrak{r}}
\newcommand{\fks}{\mathfrak{s}}
\newcommand{\fkt}{\mathfrak{t}}
\newcommand{\fkx}{\mathfrak{x}}
\newcommand{\fky}{\mathfrak{y}}
\newcommand{\fkz}{\mathfrak{z}}
\newcommand{\fkA}{\mathfrak{A}}
\newcommand{\fkE}{\mathfrak{E}}
\newcommand{\fkO}{\mathfrak{O}}
\newcommand{\fkP}{\mathfrak{P}}
\newcommand{\fkR}{\mathfrak{R}}
\newcommand{\fkT}{\mathfrak{T}}
\newcommand{\fkU}{\mathfrak{U}}
\newcommand{\fkW}{\mathfrak{W}}
\newcommand{\fkX}{\mathfrak{X}}
\newcommand{\fkY}{\mathfrak{Y}}
\newcommand{\fkZ}{\mathfrak{Z}}

%% overset
\newcommand{\tl}[1]{\widetilde{#1}}


\def\vol{{\mathrm{vol}}}
\def\sgn{{\mathrm{sgn}}}
\def\skew{{\mathrm{skew}}}
\def\cdf{{\mathrm{cdf}}}
\def\sph{{\mathrm{sph}}}

\begin{document}

\section{Idea}

The success probability estimator in \cite{PKC:PosVir21} currently, at the end of each tour, models the projection of the embedded vector $\vec v$ into the final $\beta$ dimensions as:
\[ \norm{\pi(\vec v)}^2 \sim \sigma^2 \chi^2_\beta \]

This is because, if \(\vec v\) is sampled from a discrete Gaussian, then its projection into a random $\beta$-dimensional subspace is indeed distributed as \(\sigma^2 \cdot \chi^2_{\beta}\).
However, this model then implicitly assumes that \(\vec v\) is sampled freshly at the end of every tour.

In reality, \(\vec v\) doesn't change when you're attacking a single LWE instance. We have a fixed embedded vector \(\vec v\), and at the end of each tour we are simply projecting it into a (heuristically) random subspace.
Assuming \(\norm{\vec v}\) is fixed, its projection into a random $\beta$-dimensional subspace follows
\[\norm{\pi(\vec v)}^2 \sim \norm{\vec v}^2 \cdot \mathrm{Beta}\left(\frac{\beta}{2}, \frac{N-\beta}{2}\right)\]
and we should accumulate these probabilities instead.

Note we now need to take \(\norm{\vec v}\) as input to the probability estimator.

Of course in practice $\vec v$ is randomised between LWE instances, so its length is not fixed. If we want to estimate success probability against LWE with some secret/error distribution, we can work out the distribution of \(\norm{\vec v}\) and then integrate
\[\int \mathrm{Pr}(\text{succeed given } \norm{\vec v} = t) \cdot \mathrm{Pr}(\norm{\vec v}= t) dt\]
or, given that the square norm $\norm{\vec v}^2$ can only take integer values for integer lattices, this becomes the sum
\[\sum_{x = 1}^\infty \mathrm{Pr}(\text{succeed given } \norm{\vec v}^2 = x) \cdot \mathrm{Pr}(\norm{\vec v}^2 = x)\]

\subsection{Testing the model}

We have two options to test the model (and should do both);
\begin{enumerate}
    \item Compute this sum/integral and compare with experimental data;
    \item Only compute the first term, with some fixed value for $\norm{\vec v}$, and then test the model by rejection-sampling LWE instances to ensure \(\norm{\vec v}\) is in some small desired interval
\end{enumerate}

\subsection{Advantages of the new model}
\begin{itemize}
    \item Hopefully more accurate;
    \item Allows us to capture various secret and error distributions (in \cite{PKC:PosVir21} they use the Gaussian model for all secret types); hence our model could also explain the different behaviour for different distributions (e.g.\@ the `plateaus' they describe for Gaussian).
    \item More theoretically justified
\end{itemize}

\subsection{Computing the Sum}
How do we compute this sum for different distributions on $\norm{\vec v}$? It will have to be numerically, since the first term is the output of our uSVP success probability simulator.
It remains to compute the second term for various distributions we wish to model:\footnote{Currently ignoring the fact the last entry is deterministic and treating it as random. We can either fix this (easy) or just say the different is miniscule anyway.}

\begin{itemize}
    \item \textbf{Centred Binary:} If $\vec v \in \{-1,1\}^{N}$ (uniformly) then $\norm{\vec v}^2$ is constant, so the sum consists of only one term!
    \item \textbf{Ternary:} If $\vec v \in \{-1, 0, 1\}^{N}$ (uniformly) then \[ \norm{\vec v}^2 \sim \mathrm{Bin}\left(N, \ 2/3 \right)\]
    \item \textbf{Discrete Gaussian:} While it is difficult to capture the distribution of the \emph{discrete} Gaussian exactly, note that if $\vec v$ were sampled from a \emph{continuous} Gaussian of variance $\sigma^2$ and centre $0$, then $\norm{\vec v}^2 \sim \sigma^2 \chi^2_{N}$.
    So, we could make the approximation that:
    \[ \mathrm{Pr}(\norm{\vec v}^2 = x) = \mathrm{Pr}\left(x - \frac{1}{2} \le \sigma^2 \chi^2_{N} \le x + \frac{1}{2}\right) \]
    with the tails handled accordingly.
\end{itemize}

\subsubsection{For a Ternary Distribution}
Then:
\begin{align*}
    \Pr\left[\ln \norm{\pi(\omega \vv)} = x\right]
    % &= \int_{y=0}^{+\infty} \Pr_{\vu=(u_i)_i\sim \SS^{d(m+r)}_{y}}\left[\ln \left(\sum_{i=1^{d\beta_K}}u_i^2\right) = 2x\right] \cdot \Pr\left[\norm{\vv} = y\right] dy
    % \\
    &=  \sum_{z=0}^{d(m+r)} \Pr\left[P_{\sqrt{z+d\sigma^2}} = x\right] \cdot \begin{pmatrix}
        d(r+m) \\ z
    \end{pmatrix} \frac{2^z}{3^{d(r+m)}}
\end{align*}
where:
\begin{align*}
    \Pr\left[P_{\sqrt{z+d\sigma^2}} = x\right] &=  \frac{1}{B\left(\frac{f_1}{2}; \frac{f_2}{2}\right)} \left(\frac{f_1}{f_2}\right)^{f_1-1} \left(\frac{{z+d\sigma^2}}{e^{2x}}-1\right)^{\frac{f_1}{2}-1} \left(1+\frac{f_1^2}{f_2^2}\left(\frac{{z+d\sigma^2}}{e^{2x}}-1\right)\right)^{-\frac{f_1+f_2}{2}}
\end{align*}
where $\boxed{f_1 = N-d\beta_K, f_2 = d\beta_K}$, and $B$ is the beta function.


\subsubsection{For a Discrete Gaussian $\vv$.}\paola{Copied from notes.tex in the other project}
The previous paragraph models $\norm{\vv}$ with a $\chi$-square distribution, but $\vv$'s first $d(m+r)$ coordinates will in fact be drawn from a discrete Gaussian distribution with integer coefficients and the same variance $\sigma^2$ as in the continuous model (and the last $d$ coordinates will be equal to $\sigma$). It is then natural to discuss how far away the continuous model is from the discrete one. In this more accurate discrete model, using the random variable $Y \sim D_{d(r+m), \sigma}$\paola{Discrete gaussian with $d(r+m)$ integer coefficients and standard deviation $\sigma$}:
\begin{align*}
    \Pr\left[\ln \norm{\pi(\omega \vv)} = x\right]
    % &= \int_{y=0}^{+\infty} \Pr_{\vu=(u_i)_i\sim \SS^{d(m+r)}_{y}}\left[\ln \left(\sum_{i=1^{d\beta_K}}u_i^2\right) = 2x\right] \cdot \Pr\left[\norm{\vv} = y\right] dy
    % \\
    % \approx d_{\sph}(x)
    &=  \sum_{z=0}^{+\infty} \Pr\left[P_{\sqrt{z+d\sigma^2}} = x\right] \cdot \Pr\left[\norm{Y}^2 = z\right].
    % \\
    % &= \int_{y=\sigma}^{+\infty} \Pr\left[P_y = x\right] \cdot \frac{\left(\left(\frac{y}{\sigma}\right)^2-d\right)^{\frac{r+m}{2}-1} e^{-\frac{\left(\frac{y}{\sigma}\right)^2-d}{2}}}{2^{\frac{r+m}{2}} \Gamma\left(\frac{m+r}{2}\right)}  dy
\end{align*}
Where, denoting
$s = \sigma \sqrt{2\pi}$,
$\rho(\vx) = e^{-\pi\frac{\norm{\vx}^2}{s^2}}$ and $\rho(\ZZ^{d(r+m)}) = \sum_{\vx\in\ZZ^{d(r+m)}} \rho(\vx)$:
\begin{align*}
    \Pr\left[\norm{Y}^2 = z\right] &= \sum_{\vx\in\ZZ^n \vert \norm{\vx}^2 = z} \frac{e^{-\pi \frac{z}{s^2}}}{\rho(\ZZ^{d(r+m)})}
    % \\
    % &= \frac{z^{\frac{d(r+m)-1}{2}} 2 \pi^{d(r+m)} e^{-\pi\frac{z}{\sigma^2}}}{\Gamma\left(\frac{d(r+m)}{2}\right)\rho(\ZZ^{d(r+m)})}
\end{align*}
where:
\begin{align*}
    \frac{1}{1 + \left(\frac{s}{\sqrt{\pi}}\right)^{d(r+m)}} &\leq \frac{1}{\rho\left(\ZZ^{d(r+m)}\right)} \leq 1
    \\
    \frac{1}{1 + \left(\sigma\sqrt{2}\right)^{d(r+m)}} &\leq \frac{1}{\rho\left(\ZZ^{d(r+m)}\right)} \leq 1
\end{align*}
and $\sum_{\vx\in\ZZ^n \vert \norm{\vx}^2 = z} e^{-\pi \frac{z}{s^2}}$ is the $z$-th term of $\Theta_{\ZZ^{d(r+m)}}\left(e^{-\frac{\pi}{s^2}}\right)$, where $\Theta_{\ZZ^{d(r+m)}}$ is the theta-serie of $\ZZ^{d(r+m)}$.\paola{Helpful?}

Then,
as spherical caps around one sphere point with area one are subsets of the radius-one ball around this point,
we also have:\paola{I thinl we could even prove that it is less than $\frac{1}{\pi}$ times the bound I wrote}\paola{To do: add lower bound, cf. \cite{BAMS:Olds41}: lower bound $6$ 3-representations for $z$ a power of two $\to$ lower bound of $6 \begin{pmatrix}
    n \\ 3
\end{pmatrix} = n(n-1)(n-2)$}
\begin{align*}
    \sum_{\vx\in\ZZ^n \vert \norm{x}^2 = z} \leq \int_{\SS_{\sqrt{z}}} = \frac{2\sqrt{\pi^{d(r+m)} z^{d(r+m)-1}}}{\Gamma\left(\frac{d(r+m)}{2}\right)}
\end{align*}
so finally:
\begin{align*}
    \Pr\left[\norm{Y}^2 = z\right] \leq \Pr\left[\norm{X}^2 = z\right]
\end{align*}

\paola{Forget former (commented) paragraph, was if the discrete Gaussian were a rounded one, but this is not the right approach}
% \paragraph{An Exact discrete Gaussian Model?}

% The above approximation of the distribution of the square norm of $\vv$ with discrete Gaussian coordinates is not exact, as it performs a discretisation on the square norm variable instead of the individual coordinates.
% Let us denote $f_\vv$ the cdf of the variable $\norm{\vv}^2$. For each integer $x$ we have:

% \begin{align*}
%     f_\vv(y) = \sum_{(x_i)_i \in\SS_y} \prod_{i=1}^N \Pr\left(G_i = x_i\right)
% \end{align*}
% where $\SS_{y} = \left\{(x_i)_i \in \ZZ^N \vert \sum_{i} x_i^2 = y\right\}$, and the $G_i$ are independent variables following discrete centered Gaussian distributions with standard deviation $\sigma$ and integer values.
% For any $x_i\in\ZZ$, $\Pr\left(G_i = x_i\right) = \Pr\left(N\in\left[x_i - \frac{1}{2}; x_i + \frac{1}{2}\right]\right)$, for $N$ following a centered normal distribution $\N(0;\sigma)$. Hence, $\Pr\left(G_i = x_i\right) =\frac{1}{2} \left(\erf\left(\frac{2x_i+1}{2\sigma\sqrt{2}}\right) - \erf\left(\frac{2x_i-1}{2\sigma\sqrt{2}}\right) \right)$.

% Moreover, the number of elements in $\SS_y$ is for instance studied in \cite{BAMS:Olds41}. Given $y$ of prime decomposition $2^k\left(\prod_i p_i^{a_i}\right) \left(\prod_j q_j^{b_j}\right)$, where the $p_i$ are primes equal to $1 \mod 4$ and the $q_j$ equal to $3 \mod 4$, $\abs{\SS_y} = 6\left(\prod_i p_i^{a_i}\right) \prod_j \left(q_j^{b_j} + 2\ \frac{q_j^{b_j} - 1}{q_j - 1}\right)$.
% \paola{Actually complicated to obtain something as the product of the $\erf$ terms does not lead to a simplification allowing to express it in terms of $y$, as in the continuous case. }\paola{Maybe the discrepancy could be estimated though}


\bibliographystyle{alpha}
\bibliography{cryptobib/abbrev3,cryptobib/crypto,more-bib}


\end{document}
